\begin{ejercicio}{8}{
	Sea $\pi$ una distribución invariante para la matriz de transición irreducible $P$. Probar que $\pi(x) > 0$ para todo $x \in S$ sin usar la fórmula vista en clase.
}{
	Si $\pi (x) = 0$, por ser $\pi$ distribuci\'on invariante, vale $\pi = \pi P$. Luego
	\[
		0 =
		\pi(x)
		= (\pi P)(x)
		= \sum_{y \in \Omega} \pi(y) P(y,x)
	\]
	Como todos los terminos corresponden a probabilidades, son no negativos, luego
	\begin{equation}
		\label{L}
		\pi(y) P(y,x) = 0 \quad \forall y \in \Omega
	\end{equation}

	Como $\pi$ es una distribuci\'on, $\sum_{y \in \Omega} \pi(y) = 1$, y por lo tanto existe un
	$y \in \Omega$ tal que $\pi(y) \neq 0$.


	Como $P$ es irreducible, $\exists r \in \mathbb N / P^r(y,x) > 0$.


	Luego
	\begin{math}
		\exists z_1, \dots, z_{r-1} \in \Omega
	/
		P(y,z_1) P(z_1, z_2) \dots P(z_{r-2}, z_{r-1}) P(z_{r-1}, x) >0
	\end{math}


	Tenemos que valen $P(z_{r-1}, x) > 0$, $\pi(x) = 0$,
	y usando (\ref{L}) vale
	$\pi(z_{r-1}) P(z_{r-1}, x) = 0$.
	Luego $\pi(z_{r-1}) = 0$.


	Similarmente,
	\begin{align*}
		P(z_{r-2}, z_{r-1}) > 0, \pi(z_{r-1}) = 0 &\implies \pi(z_{r-2}) = 0 \\
		&\vdots \\
		P(z_1, y) > 0, \pi(z_1) = 0 &\implies \pi(y) = 0
	\end{align*}
	Pero $\pi(y) \neq 0$
}
	
\end{ejercicio}