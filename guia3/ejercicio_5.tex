\paragraph{Ejercicio 5}

En 1975 se condujo un experimento para ver si la siembra de nubes produc\'ia lluvia.
Se sembraron 26 nubes con nitrato de plata y otras 26 se dejaron sin sembrar.
La elecci\'on de sembrar o no cada una de las nubes fue hecha al azar.
Los datos del experimento se encuentran en
\texttt{https://www.stat.cmu.edu/\~larry/all-of-nonpar/data.html}

Sea $\theta = T(F_1) - T(F_2)$ la diferencia entre la mediana de las precipitaciones de cada uno de los grupos.
Estimar $\theta$.

\paragraph{$D/$}
	$F^{-1}(\frac 1 2) = \inf \{x \in \mathbb R : F(x) \ge \frac 1 2\}$ es una mediana de $F$.

	En el caso de una distribuci\'on emp\'irica $\hat F_n$, buscar este \'infimo es buscar el m\'inimo $x$ tal que
	\begin{gather*}
		\hat F_n(x) = \frac 1 n \sum_{i=1}^{n} \bbone \{ X_i \le x \} \ge \frac 1 2
	\end{gather*}
	Esto vale si y solo si
	\begin{gather*}
		\sum_{i=1}^{n} \bbone \{ X_i \le x \} \ge \frac n 2
	\end{gather*}
	Que se minimiza cuando
	\begin{gather*}
		\sum_{i=1}^{n} \bbone \{ X_i \le x \} = \floor*{\frac {n+1} 2}
	\end{gather*}

	Luego,
	si $X^{(1)} \le \dots \le X^{(n)}$
	corresponde a las variables
	$X_1, \cdots, X_n$
	ordenadas de manera ascendente,
	$\hat F_n^{-1}(\frac 1 2) = X^{(\floor*{\frac {n+1} 2})}$
	es un estimador plug-in de la mediana.

	El siguiente script de Python, con el archivo \verb|clouds.dat|
	en el directorio donde se ejecuta el script,
	calcula esa mediana para los datos que se solicitan
\begin{verbatim}
import csv

data = []
with open('clouds.dat') as f:
    reader = csv.DictReader(f, delimiter='\t')
    for row in reader:
        data.append(row)

keys = data[0].keys()
data = {k: [float(data[i][k]) for i in range(len(data))] for k in keys}

def median(l):
    return sorted(l)[ (len(l) + 1) // 2 ]

print ({k: median(data[k]) for k in keys})
\end{verbatim}

Al finalizar muestra en la consola
\begin{verbatim}{'Unseeded_Clouds': 47.3, 'Seeded_Clouds': 242.5}\end{verbatim}

Luego $\hat \theta = 242.5 - 47.3 = 195.2 \text{aft} = 240775.3 \text{m}^3$ si $F_1$ correspondia a las semillas sembradas, y $F_2$ a aquellas que quedaron sin sembrar.
\qed