\documentclass[10pt,a4paper]{article}
\usepackage{amsmath}
\usepackage{amssymb}

\newcommand{\R}{ \mathbb R }
\newcommand{\E}{ \mathbb E }
% \newcommand{\S}{ \mathbb S }

\newenvironment{ejercicio}[3]
{
	\paragraph{Ejercicio #1}
	#2
	\paragraph{$D/$}
	#3
	\begin{flushright}$\square$\end{flushright}
}{
}



\begin{document}


\begin{ejercicio}{5}{
	Sea $f:[0,1] \to \R$ una funci\'on continua. Probar que
	$$
		\lim_{d\to\infty}
		\int_0^1 \dots \int_0^1
		f\left(
		\frac{
			x_1 + \dots + x_d
		}{d}
		\right)
		dx_1 \dots dx_d
		=
		f\left(\frac{1}{2}\right)
	$$
}{
	Sean $(U_d)_{d \in \mathbb N}$ v.a.i.i.d. con distribucion uniforme en $[0,1]$

	Por ley de grandes numeros,
	$$
		\frac {U_1 + \dots + U_d}{d} \overset{P}{\to} \frac{1}{2}
	$$

	Como $f$ es continua,
	$$
		f\left(\frac {U_1 + \dots + U_d}{d}\right) \overset{P}{\to} f\left(\frac{1}{2}\right)
	$$

	Por ser $f$ continua en el compacto $[0,1]$, $f([0,1])$ es acotada y, en particular, las variables aleatorias $f(\frac {U_1 + \dots + U_d}{d}) \subseteq f([0,1])$ estan acotadas para todo $d \in \mathbb N$, luego
	$$
		f\left(\frac {U_1 + \dots + U_d}{d}\right) \overset{L^1}{\to} f\left(\frac{1}{2}\right)
	$$


	Tomando esperanza de ambos lados queda el resultado pues $\E[f(\frac{1}{2})] = f(\frac{1}{2})$, y
	$\E[f(\frac {U_1 + \dots + U_d}{d})]
	=
	\int_0^1 \dots \int_0^1
	f(
	\frac{
		x_1 + \dots + x_d
	}{d}
	)
	dx_1 \dots dx_d$

}
\end{ejercicio}


\begin{ejercicio}{8a}{
	Probar que si $\Phi$ es convexa,
	$$
		\Phi(y) \ge \Phi(x) + \Phi'(x)(y-x)
	$$
	para todo $ x,y \in \R $
}{
	Sea $x \in \R$, considero el polinomio de taylor de $\Phi$ de grado 2 centrado en $x$, luego para todo $y \in \R$, existe $\xi$ tal que:
	$$
		\Phi(y) = \Phi(x) + \Phi'(x) (y-x) + \Phi''(\xi) \frac{(y-x)^2}{2}
	$$
	$\Phi(\xi) \ge 0$ por hipotesis, luego
	$$
		\Phi(y) \ge \Phi(x) + \Phi'(x) (y-x)
	$$
}
\end{ejercicio}

\begin{ejercicio}{8b}{
	Probar que 
	$$
		\Phi(\frac{x+y}{2}) \le \frac{1}{2}\Phi(x) + \frac{1}{2}\Phi(y)
	$$
	para todo $ x,y \in \R $
}{
	Aplicando el item anterior,
	$$
		\Phi(x) \ge \Phi(\frac{x+y}{2}) + \Phi'(\frac{x+y}{2}) \frac{x-y}{2}
	$$
	y
	$$
		\Phi(y) \ge \Phi(\frac{x+y}{2}) + \Phi'(\frac{x+y}{2}) \frac{y-x}{2}
	$$

	Sumando las desigualdades queda
	$$
		\Phi(x) + \Phi(y) \ge 2 \Phi\left(\frac{x+y}{2}\right)
	$$
}
\end{ejercicio}

\begin{ejercicio}{8c}{
	$\Phi: \R^d \to \R$	se dice convexa si la matriz $(\Phi_{x_i x_j})$ es semidefinida positiva para todo $x \in \R^d$.
	Es decir, $\sum_{i=1}^n \sum_{j=1}^n \Phi_{x_i x_j} \xi_i \xi_j \ge 0$ para todo $\xi \in \R^d$.
	Probar que para tal $\Phi$,
	\begin{equation}
	\label{cota}
		\Phi(y) \ge \Phi(x) + \nabla\Phi(x) \cdot (y-x)
	\end{equation}
	y
	$$
		\Phi(\frac{x+y}{2}) \le \frac{1}{2}\Phi(x) + \frac{1}{2}\Phi(y)
	$$
}{
	Considero el polinomio de Taylor de $\Phi$ alrededor de $x \in R^d$
	$$
		\Phi(y) = \Phi(x)
		+ \sum_{i=1}^d \frac {\partial \Phi}{\partial x_i}(x) (y_i - x_i)
		+ \sum_{i=1}^d \sum_{j=1}^d \frac{\partial^2 \Phi}{\partial x_i \partial x_j}(x) (y_i - x_i) (y_j - x_j)
	$$

	el t\'ermino $\sum_{i=1}^d \sum_{j=1}^d \frac{\partial^2 \Phi}{\partial x_i \partial x_j}(x) (y_i - x_i) (y_j - x_j)$ es $\ge 0$ por hipotesis,

	y $\nabla\Phi(x) \cdot (y-x) = \sum_{i=1}^d \frac {\partial \Phi}{\partial x_i}(x) (y_i - x_i)$.
	Luego vale (\ref{cota})

	Para ver la segunda parte, aplicamos (\ref{cota})
	$$
		\Phi(x) \ge \Phi\left(\frac{x+y}{2}\right) + \nabla\Phi\left(\frac{x+y}{2}\right) \cdot \frac{x-y}{2}
	$$
	$$
		\Phi(y) \ge \Phi\left(\frac{x+y}{2}\right) + \nabla\Phi\left(\frac{x+y}{2}\right) \cdot \frac{y-x}{2}
	$$

	Sumando las desigualdades queda
	$$
		\Phi(x) + \Phi(y) \ge 2 \Phi\left(\frac{x+y}{2}\right)
	$$
}
\end{ejercicio}

\begin{ejercicio}{9}{
	Probar la desigualdad de Jensen
	$$
		\Phi(\E(X)) \le \E(\Phi(X))
	$$
	para toda variable aleatoria $X$ y $\Phi$ convexa
}{
	Supongamos $\E(X) < \infty$, y $\E (\Phi(X)) < \infty$.

		Vale $ \Phi(X) \ge \Phi(\E(X)) + \Phi'(\E(X))(X - \E(X)) $ c.s. :
		Pues fijando un $\omega \in \Omega$, por el ejercicio 8a,
		$$
			\Phi(X(\omega))
			\ge
			\Phi(\E(X)) + \Phi'(\E(X)) (X(\omega) - \E(X))
		$$

		Luego, por monotonía de la esperanza
		$$
			\E[\Phi(X)]
			\ge
			\E[\Phi(\E(X)) + \Phi'(\E(X)) (X - \E(X))]
			$$$$
			=
			\Phi(\E(X)) + \Phi'(\E(X)) \E[X - \E(X)]
			=
			\Phi(\E(X))
		$$
}\end{ejercicio}


\begin{ejercicio}{10}{
	Sea $s_d$ el vol\'umen de la esfera de radio uno $\mathbb S^{d-1} = \{x \in \R^d, |x| = 1\}$ de $\R^d$.
	Probar que el vol\'umen de la esfera de radio $r$ en $\R^d$ es $s_d r^{d-1}$.
}{
	Sea $R>0$.

	Haciendo el cambio de variable $T: x \in \R^d \mapsto Rx \in \R^d$, que tiene determinante $R^d$,
	tenemos que $vol(B_R^d) = R^d vol(B^d)$:
	$$
		vol(B_R^d)
		=
		\int_{B_R^d} 1 dV
		=
		\int_{B_1^d} R^d dV
		=
		R^d vol(B^d)
	$$

	Como la bola de radio $R$ de $\R^d$ es unión disjunta de las esferas de radio $r$ para $0 \le r \le R$, vale
	$$
	vol(B_R^d)
	=
	\int_{B_R^d} 1 dV
	=
	\int_0^R vol(\mathbb S_r^{d-1}) dr
	$$

	La función $R \mapsto vol(B_R^d)$ es diferenciable por ser una integral, luego

	$$
		\dfrac{d}{dR} vol(B_R^d)
		=
		vol(\mathbb S_R^d) - vol(\mathbb S_0^d)
	$$

	$\mathbb S_0^d = {0}$ tiene medida cero. Por otro lado
	$$
	vol(\mathbb S_R^d)
	=
	\dfrac{\text d}{\text d R} vol(B_R^d)
	=
	\dfrac{\text d}{\text dR} R^d vol(B^d)
	=
	d \cdot R^{d-1} vol(B^d)
	=
	R^{d-1} s_d
	$$

	Donde la ultima igualdad sale de tomar $R=1$ en la ecuacion:
	$$
	s_d
	=
	vol(\mathbb S_1^d)
	=
	d \cdot vol(B^d)
	$$

}\end{ejercicio}


\begin{ejercicio}{15}{
	Sea
	$X \sim N(0,1)$
	Probar que
	$
	\E(e^{sX^2}) = \frac{1}{\sqrt{1 - 2s}}
	$ para 
	$
	-\infty < s < \frac{1}{2}
	$.
}{
	$$
		\E(e^{sX^2})
		=
		\int_\R e^{sx^2} \frac{1}{\sqrt{2\pi}} e^{-\frac{x^2}{2}} dx
	$$

	Como $sx^2 - \frac{x^2}{2} = \frac{x^2}{2}(2s-1)$, tomando el cambio de variable $u = x\sqrt{1-2s}$ tenemos
	$$
		\E(e^{sX^2})
		=
		\frac{1}{\sqrt{1 - 2s}}
		\int_\R 
		\frac{1}{\sqrt{2\pi}}
		e^{-\frac{u^2}{2}} du	
	$$
	
	y esta ultima integral eval\'ua 1 por ser densidad de una normal (0,1) 

}\end{ejercicio}

\end{document}